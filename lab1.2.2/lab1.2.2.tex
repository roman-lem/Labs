\documentclass[a4paper, 12pt]{article}

\usepackage{cmap}
\usepackage[T2A]{fontenc}
\usepackage[utf8]{inputenc}

\usepackage{amsmath, amsfonts, amssymb, amsthm, mathtools}
\usepackage{icomma}

\title{Отчет о выполнении лабораторной работы \\ Эксперементадльная проверка закона вращательного движения на крестообразном маятнике}
\author{Лепарский Роман}
\date{\today}

\begin{document}

\maketitle

\newpage

\section{Аннотация}

Целью работы является получение зависимости углового ускорения от момента прикладываемых к маятнику сил. Необходимо убедиться, что угловое ускорение зависит от момента сил линейно, определить момент инерции маятника. Также, нужно проанализировать влияние сил трения, действующих на ось вращения.

\section{Теоретические сведения}

В данной работе эксперементально поверяется уравнение вращательного движения:
\begin{equation} \label{eq:main}
I\frac{d\omega}{dt} = M
\end{equation}

Для этого используется крестообразный маятник

\begin{center}
\includegraphics[scale = 0.3]{"pendulum.eps"}
\end{center}

Массы грузов:
\begin{center}
\begin{tabular}{|l|l|}
\hline
$m_1$, г & 155,5 \\ \hline
$m_2$, г & 148,9 \\ \hline
$m_3$, г & 151,9 \\ \hline
$m_4$, г & 150,1 \\ \hline
\end{tabular}
\end{center}

\[
m_0 = <m> = \frac{1}{N}\sum_{i=1}^N m_i = \frac{155,5+148,9+151,9+150,1}{4} = 151,6\text{г} 
\]

Запишем также некоторые полезные данные:
\begin{itemize}
	\item Высота опускания груза $H = 60$см
	\item Радиус маленького шкива $r_1 = 9$мм
	\item Радиус большого шкива $r_2 = 17,5$мм
	\item Расстояние от оси вращения до центров масс грузов в 1 опыте $R_1 = 60 + 12,5 = 72,5$мм
	\item Расстояние от оси вращения до центров масс грузов во 2 опыте $R_1 = 200 + 12,5 = 212,5$мм
\end{itemize}

Вращающий момент задаётся силой натяжения $T$:
\begin{equation} \label{eq:moment}
M_H = rT
\end{equation}
где $r$ - радиус шкива. Силу $T$ легко найти из уравнения движения платформы с перегрузком:
\begin{equation} \label{eq:newton}
mg - T = ma
\end{equation}
здесь $m$ - масса платформы с перегрузком

Если момент трения в подшипниках мал по сравнению с моментом $M_T$, то из (\ref{eq:main}), (\ref{eq:moment}) и (\ref{eq:newton}) следует постоянство ускорения $a$, и, измеряя время $t$, в течение которого нагруженная платформа из состояния покоя опускается на расстояние $h$,
можно найти её ускорение $a$:
\[
a = \frac{2h}{t^2}
\]
связанное с угловым ускорением $\beta = d\omega /dt$ соотношением:
\begin{equation} \label{eq:acel}
a = r\frac{d\omega}{dt} = r\beta
\end{equation}

Для дальнейшей работы удобно преобразовать уравнение (\ref{eq:main}), выделив момент сил трения в явном виде:
\[
M_H - M_T = I\frac{d\omega}{dt}
\]



\section{Приборы и материалы}

\section{Обработка результатов}

\section{Вывод}

\end{document}